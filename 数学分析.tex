\documentclass[12pt,a4paper]{report}
\usepackage[UTF8]{ctex}
\usepackage{graphicx}
\usepackage{amsmath} 
\usepackage{amssymb} 
\usepackage{amsfonts}
\usepackage{physics}
\usepackage{pifont}
\usepackage{wrapfig}
\usepackage[final=true,hidelinks]{hyperref}
\begin{document}
    \title{数学分析\thanks{This note was made by \LaTeX \quad Final edition \today}}
    \date{2022/7/10}
    \author{Mfree}
    \maketitle
\tableofcontents{}
\chapter{序列}
\paragraph{斯托尔茨(O.Stolz)定理}
    为确定$\frac{\infty}{\infty}$型的不定式$\frac{x_n}{y_n}$的极限\\
    设整序变量$y_n\rightarrow +\infty$,并且至少是从某一项开始,在$n$增大时$y_n$也增大:
    $$y_{n+1} >y_n$$
    则
    $$\lim\frac{x_n}{y_n} =\lim\frac{x_n-x_{n-1}}{y_n-y_{n-1}}$$
\paragraph{定理}
    已给单调增大的整序变量$x_n$.若有上确界:
    $$x_n \leq M$$ 则必有一有限的极限,否则趋向$+\infty$同样的,单调减小也有类似性质
\paragraph{数$e$}
    这里使用极限步骤来定义一个新的数:$e$\\
    \quad 考察整序变量$$x_n=\left(1+\frac{1}{n}\right)^{n}$$
    使用二项式定理展开可证明$x_n$时单调递增数列
    \begin{align*}
        x_n=&1+1+\frac{1}{2!}\left(1-\frac{1}{n}\right)+\frac{1}{3!}\left(1-\frac{1}{n}\right)\left(1-\frac{2}{n}\right)+\dots \\
            &+\frac{1}{k!}\left(1-\frac{1}{n}\right)\dots \left((1-\frac{k-1}{n}\right)+\dots \\
            &+\frac{1}{n!}\left(1-\frac{1}{n}\right)\dots \left(1-\frac{n-1}{n}\right)
    \end{align*}
    证明上有界.在上式中略去括号的项
    $$x_n<2+\frac{1}{2!}+\frac{1}{3!}+\dots \frac{1}{n!}$$
    更进一步,将分母每一因子全改为2:
    $$x_n<2+\frac{1}{2}+\frac{1}{2^2}+\dots +\frac{1}{2^{n-1}}<3$$
    所以$x_n$有极限,记为:
    $$e=\lim \left(1+\frac{1}{n}\right)^n$$
    $$e=2.718281828459045\dots$$
\paragraph{区间套引理}
    给定单调增大的整序变量$x_n$及单调减小的整序变量$y_n$,且恒有$$x_n<y_n$$
    若其差$(y_n-x_n) \rightarrow 0$, 则二整序变量必有公共的有限极限:
    $$c=\lim x_n=\lim y_n$$
\section{收敛原理·部分极限}
\paragraph{收敛原理} 
设给定整序变量$x_n$,依数列
\begin{equation}
    x_1,x_2,\dots,x_n,\dots
    \label{有界序列}
\end{equation}
而递变,我们要研究序列极限的存在性的一般判定法,所以我们不能用极限定义的本身,因为极限定义中已经用到了这个极限,我们需要仅用这数列来判定极限存在性问题\par
    \paragraph*{定理} 整序变量$x_n$有有限极限的充分必要条件为:对于每一个数$\varepsilon$, \\
    总存在着序号$N$,使$n>N$及$n'>N$时,便能成立不等式
    $$\left\lvert x_n-x'_n\right\rvert <\varepsilon$$
    \paragraph{证明} 证明之前我们们
\paragraph{部分数列与部分极限}
今我们从\eqref{有界序列}中选出一部分
\begin{equation}
    x_{n_1},x_{n_2},\dots,x_{n_k},\dots
    \label{部分序列}
\end{equation}
其中$\{n_k\}$为某一自然数的递增序列\par
若数列\eqref{有界序列}有确定极限,则部分数列\eqref{部分序列}也必有相同的极限\par
事实上,不论整序变量$n_k$依什么规律趋向$+\infty$,我们的命题仍有效\par 
6
\newpage
\chapter{微分学}
\section{一元函数}
\subsection{函数}
如果依某一法则或规律,对$\mathcal{X}$中的每一$x$值总有一个确定的数值$y \in \mathcal{Y} $和它对应,则变量$y$就称为变量$x$的函数
\subsection{几类重要函数的分类}
\begin{enumerate}
    \item 有理整函数及分式函数
        $$y=a_nx^n+a_{n-1}x^{n-1}+\dots+a_1x+a_0$$
        $$y=\frac{a_nx^n+a_{n-1}x^{n-1}+\dots+a_1x+a_0}{b_mx^m+b_{m-1}x^{m-1}+\dots+b_1x+b_0}$$
    \item 幂函数$$y=x^{\mu}$$
    \item 指数函数$$y=a^x,\quad a\geqslant 0\,\mathrm{and}\,\neq 1$$
    \item 对数函数$$y=\log_a x,\quad a\geqslant 0\,\mathrm{and}\,\neq 1\,\mathrm{and}\,x\geqslant 0$$
\end{enumerate}

\subsection{连续函数的性质}
\paragraph{布尔查诺-柯西第一定理}
函数$f(x)$在闭区间上定义且连续,若在区间两端取异号。则在区间内必有一点$c$使$f(c)=0$
\paragraph{布尔查诺方法}
将区间对半分,保留区间两端上的函数值取异号的区间,重复上述操作,对第n个区间$[a_n,b_n]$,有
\begin{equation}
f(a_n)<0,\quad f(b_n)>0
\label{两端点不等式}
\end{equation}
它的区间长度
$$
b_n-a_n=\frac{b-a}{2^n}
$$
显然,随着n的趋向$+\infty$有$\lim (b_n-a_n) =0$,因此在区间[a,b]中存在点c,满足
$$
\lim a_n =\lim b_n =c
$$
将不等式\eqref{两端点不等式}取极限。应用函数的连续性
$$
f(c)=\lim f(a_n) \leqslant 0 \quad and \quad f(c)=\lim f(b_n) \geqslant 0
$$
则必有$f(c)=0$

\chapter{积分学}
\section{黎曼和}
\paragraph{定义}
设函数$f(x)$在给定区间$[a,b]$上,用任意方法在$a$和$b$之间插入分点
$$
x_0=a,x_1,x_2,\cdots,x_{n-1},x_n=b
$$
把区间分成若干部分,使用$\lambda$来表示差$\Delta x_i=x_{i+1}-x_i (i=1,2,\cdots,n-1)$
中最大的部分

从各区间中任取一点$x=\xi_{i}$
$$x_i \leq \xi_i \leq x_{i+1} \quad (i=1,2,\cdots,n-1)$$
并做和
$$
\sigma=\sum_{i=1}^{n-1} f(\xi_i) \Delta x_i
$$
若
$$
\lim_{\lambda \rightarrow 0} \sigma =I
$$
成立,我们称该极限为$f(x)$在$a$到$b$的区间上的定积分,并记为
$$
I=\int_a^b f(x) \, \mathrm{d}x
$$
$f(x)$叫做区间$[a,b]$上的\textbf{可积函数}
\chapter{级数}
\paragraph{定义}
设给定某一无穷序列
$$
a_1,a_2,\cdots,a_n,\cdots
$$
这些数的和称为无穷级数,记为:
$$
\sum_{n=1}^{\infty} a_n =a_1+a_2+\cdots+a_n+\cdots
$$

\paragraph{部分和}
若弃去级数前$m$个项,所留下的剩余的项的和为级数第$m$项后的\textbf{余式}
$$
    A_m=\sum_{n=m+1}^{\infty} a_n=\sum_{n=1}^{\infty} a_n - \sum^{m} a_n
$$
\end{document}
